\documentclass{article}

\usepackage{amsmath}
\usepackage{amssymb}
\usepackage{amsthm}
\usepackage{mathrsfs}
\usepackage[english]{babel}

\newtheorem{theorem}{Theorem}[section]

\theoremstyle{definition}
\newtheorem{definition}{Definition}[section]


\newtheorem*{remark}{Remark}

\newcommand{\R}{$\mathbb{R}$}

\newcommand{\m}[1]{$ #1 $}

\begin{document}
	

\begin{titlepage}
   \vspace*{\stretch{1.0}}
   \begin{center}
      \Large\textbf{Measure Theory Notes}\\
      \large\textit{Cheung Hau Wa}
   \end{center}
   \vspace*{\stretch{2.0}}
\end{titlepage}
\section{Basics}
\begin{remark}
	I will directly start with \textit{Borel $\sigma$-algebra} because that is where things get compilcated
\end{remark}
\begin{definition}
	The $\sigma$-algebra $\sigma(\mathscr{O})$ generated by the open sets $\mathscr{O}$ of $\mathbb{R}^n$ is called \textit{Borel $\sigma$-algebra}
\end{definition}
\begin{remark}
	For every system of sets $\mathscr{G} \subset \mathscr{P}(X)$ there exists a smallest $\sigma$-algebra containing $\mathscr{G}$. This is the \textit{$\sigma$-algebra generated by $\mathscr{G}$} which is just \begin{equation*}
		\mathscr{A} = \bigcap_{\substack{\mathscr{F} \text{ }\sigma\text{-alg} \\ \mathscr{F} \supset \mathscr{G}}} \mathscr{F}
	\end{equation*}
\end{remark}
	
Since Borel sets are fundamental for both measure theory and topology, we consider the following amaizng theorem
\begin{theorem}
	Let $\mathscr{O,C,K}$ be families of open, closed, comapct sets in $\mathbb{R}^n$.Then 
	\begin{equation*}
		\mathscr{B}(\mathbb{R}^n) = \sigma(\mathscr{O}) = \sigma(\mathscr{C}) = \sigma(\mathscr{K})
	\end{equation*}
\end{theorem}
\begin{proof}
	Since compact sets are closed, we have $\mathscr{K} \subset \mathscr{C}$ , hence $\sigma(\mathscr{K}) \subset \sigma(\mathscr{C})$. On the other hand, if $C \in \mathscr{C} \text{, then } C_k = C \cap \overline{B_k(0)} $ is closed and bounded, hence $C_k \in \mathscr{K}$. Notice \m{C = \bigcup_{k \in \mathbb{N} C_k}},thus \m{\mathscr{C} \subset \sigma(\mathscr{K})}, hence \m{\sigma(\mathscr{C)} \subset \sigma(\mathscr{K})}

	Since \m{(\mathscr{O})^c=\mathscr{C}} we have \m{\mathscr{C} \subset \sigma(\mathscr{O})}.
	And the converse is similar
\end{proof}

Notice from our above theorem, a lot of unexpected(at least for me) sets can be a Borel \m{\sigma}-algebra. Consider: 
\begin{align*}
	\mathscr{J}^{o}(\mathbb{R})^{n} = \{(a_i,b_i):a_i,b_i \in \mathbb{R}, i \in \mathbb{N}\} \,
        \mathscr{J}(\mathbb{R})^{n} = \{[a_i,b_i):a_i,b_i \in \mathbb{R}, i \in \mathbb{N}\} 
\end{align*}
For for notation, we write \m{\mathscr{J}_{rat},\mathscr{J}^{o}_{rat}} as (half-)open interval with rational endpoints. For which we have the following theorem
\begin{theorem}
	\m{\mathscr{B}(\mathbb{R})^n = \sigma(\mathscr{J}^{n}_{rat}) =  \sigma(\mathscr{J}^{o,n}_{rat}) =  \sigma(\mathscr{J}^{n}) =  \sigma(\mathscr{J}^{n,o}) }
\end{theorem}
\begin{proof}
	Consider an obvious fact: \m{\sigma(\mathscr{O}) \supset \sigma(\mathscr{J}^{o}) \subset \sigma(\mathscr{J}^{o}_{rat}) }. For converse direction, if \m{U \in \mathscr{O}}, we have 
	\begin{equation*}
		U = \bigcup_{\substack{I \in \mathscr{J}^{o}_{rat} \\ I \subset U}} I
	\end{equation*}
	The \m{ \supset} direction is obvious, for the other direction we fix some \m{x \in U}. Since \m{U} is open, there is some ball \m{ B_{\epsilon}(x) \subset U} and we can inscribe a square into the ball \m{ B_{\epsilon}(x)} and shrink this square to get a rectangle \m{ I' = I' \in \mathscr{J}^o_{rat}} containing \m{x}. So \m{U \subset I} for all \m{I}
\end{proof}
\end{document}
