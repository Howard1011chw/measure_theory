\documentclass{article}

\usepackage{amssymb}
\usepackage{amsmath}
\usepackage{amsthm}
\usepackage{mathrsfs}
\usepackage{enumerate}
\usepackage{mathtools}

\theoremstyle{definition}
\newtheorem{solution}{Solution}

\begin{document}
\section*{Sample Solution}
\begin{solution}
	\begin{enumerate}[i)]
\item			let $A_1,\dots,A_n \in \mathscr{A}$, and then by definition $\bigcup_{1 \le i \le n} A_n \in \mathscr{A}$, since $A_i^c \in \mathscr{A} $ for all $i$. Hence,
	\begin{equation*}
		\bigcup{A_i^c} = (\bigcap{A_i})^c \in \mathscr{A}
	\end{equation*}
	implies 
	\begin{equation*}
		\bigcap{A_i} \in \mathscr{A}
	\end{equation*}
	where $i \in \{1,\dots,n\}$
\item Just use the complement axiom twice
\item Notice that $A\backslash B = A \cap B^c$ and $A \triangle B = (A \backslash B) \cup (B \backslash A)$ then it is obvious
	
\end{enumerate}	
\end{solution}
	
\vspace{1em}

\begin{solution}
	It would be crazy to write to write down exactly what is contained in it, which is the case for all this kind of objects that has the well-known property of smallest and intersection. But we can at least make the following observation \\
	Since every singleton is a Borel $\sigma$-algebra, $\sigma(A) \subset \sigma(\mathcal{O})$. Same observation can be made from the fact that $A \subset \mathcal{O}$ hence  $\sigma(A) \subset \sigma(\mathcal{O})$(exercise). \\
	However  $\sigma(A) \not \supset \sigma(\mathcal{O})$. Since countable unions (or intersections) of countable sets are countable, so any $X \in \sigma(A)$ is either countable or compliment of a countable set. Then consider $(0,1) \not \in \sigma(A)$ but is a Borel set.
\end{solution}
\end{document}
